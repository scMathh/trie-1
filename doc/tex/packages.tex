\usepackage{syntax}
\usepackage[brazilian]{babel}
\usepackage[utf8]{inputenc}
\usepackage[light,condensed,math]{iwona}
\usepackage[T1]{fontenc}
% \usepackage[usenames, dvipsnames]{xcolor}
\usepackage[usenames, dvipsnames]{color}
% \usepackage{tgtermes} times homan font similar
\usepackage{hyperref}
\usepackage{indentfirst}
\usepackage{bbding}
% \usepackage{pifont}
\usepackage{makeidx}
\makeindex
\hypersetup{
    colorlinks,
    citecolor=black,
    filecolor=black,
    linkcolor=black,
    urlcolor=black
}
\usepackage{fullpage}
\usepackage{amssymb}
\usepackage{float}
\usepackage[toc,page]{appendix}
\usepackage{cite}
% \usepackage{draftwatermark}
% \SetWatermarkText{RASCUNHO}
% \SetWatermarkScale{1.0}
% \SetWatermarkColor[rgb]{0.9, 0.9, 0.9}

% \setdefaultlanguage[babelshorthands]{brazilian}
% \usepackage{fontspec}
% Pra mostra codigo fonte
\usepackage{listings}
\usepackage{caption}
\usepackage[most]{tcolorbox}
\DeclareCaptionFont{white}{\color{white}}
\DeclareCaptionFormat{listing}{%
  \parbox{\textwidth}{\colorbox{gray}{\parbox{\textwidth}{#1#2#3}}\vskip-4pt}}
\captionsetup[lstlisting]{format=listing,labelfont=white,textfont=white}
\lstset{frame=lrb,xleftmargin=\fboxsep,xrightmargin=-\fboxsep}

%%%  Definiciao da sintaxe da linguagem jSmall para o hiligth
\lstdefinelanguage{jSmall}{
    keywords={numfi, numf, true, false, loop, return, if, in, while, then, else, elif},
    keywordstyle=\color{blue}\bfseries,
    ndkeywords={boot, export, boolean, throw, implements, import, this},
    ndkeywordstyle=\color{darkgray}\bfseries,
    identifierstyle=\color{black},
    sensitive=false,
    comment=[l]{~},
    morecomment=[s]{~\{}{\}~},
    commentstyle=\color{purple}\ttfamily,
    stringstyle=\color{red}\ttfamily,
    morestring=[b]',
    morestring=[b]"
}

\lstset{language=C++,
    basicstyle=\ttfamily,
    keywordstyle=\color{blue}\ttfamily,
    stringstyle=\color{red}\ttfamily,
    commentstyle=\color{green}\ttfamily,
    morecomment=[l][\color{magenta}]{\#}
}

\newtcblisting[auto counter]{sexylisting}[2][]{sharp corners, 
    fonttitle=\bfseries, colframe=gray, listing only, 
    listing options={basicstyle=\ttfamily,language=jSmall, numbers=left}, 
    title=Listing \thetcbcounter: #2, #1}

\newtcblisting[auto counter]{sexylistingjava}[2][]{sharp corners, 
    fonttitle=\bfseries, colframe=gray, listing only, 
    listing options={basicstyle=\ttfamily,language=java, numbers=left}, 
    title=Listing \thetcbcounter: #2, #1}
    
\newtcblisting[auto counter]{sexylistingcpp}[2][]{sharp corners, 
    fonttitle=\bfseries, colframe=gray, listing only, 
    listing options={basicstyle=\ttfamily,language=C++, numbers=left}, 
    title=Listing \thetcbcounter: #2, #1}

% Automata packages
\usepackage{tikz, graphicx}

\usepackage{pgfplots}
\pgfplotsset{compat=1.16}
\usetikzlibrary{tikzmark, shapes.callouts}
\usetikzlibrary{automata, positioning, arrows}
\usetikzlibrary{arrows.meta, % if the figure contains arrow-tips
                bending,     % arrow tips on arcs are "bent," i.e., deformed a bit
                patterns     % if the figure contains pattern fills
               }

\usepackage{tipa}

% Hacking pra poder usar syntax package junto com o tikz
\usepackage{etoolbox}
\AtBeginEnvironment{tikzpicture}{\catcode`\_=8}

\usepackage{ifthen,xcolor,xkeyval,calc}
\newlength{\tabcont}

\newcommand{\tab}[1]{%
\settowidth{\tabcont}{#1}%
\ifthenelse{\lengthtest{\tabcont < .25\linewidth}}%
{\makebox[.25\linewidth][l]{#1}\ignorespaces}%
{\makebox[.5\linewidth][l]{#1}\ignorespaces}%
}%

\frenchspacing

% \title{
%     \textbf{\Large UNIVERSIDADE FEDERAL DE ALAGOAS}\\
%     \textbf{\Large INSTITUTO DE COMPUTAÇÃO}\\
%     {\ }\\
%     \textbf{\Large Slam Combinando}\\
%     \textbf{\Large com Filtros de Kalman}\\
%     \textbf{\Large para Robôs Móveis}\\
%     % \line(1,0){250} \\
% }
\author{Joilnen Leite\\ 2017.2}

% \usepackage[colorinlistoftodos]{todonotes}\setlength{\marginparwidth}{3cm}\reversemarginpar
\usepackage {todonotes}\setlength{\marginparwidth}{3cm}\reversemarginpar

% HACK: set length so that the paper can have better width for margin

\usepackage{float}
